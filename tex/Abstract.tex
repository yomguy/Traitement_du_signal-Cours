			Ce cours enseign� au Conservatoire National des Arts et M�tiers (CNAM) de Paris est destin� � introduire les notions th�oriques et pratiques du traitement du signal � un niveau Bac +2 ou +3.

Trois parties se distinguent pour s�parer tout d'abord les �l�ments th�oriques qui fondent la \textbf{th�orie du traitement de signal} par les m�thodes de Fourier, puis une synth�se des diff�rents \textbf{modes de repr�sentation des signaux} notamment celui du spectrogramme, et enfin une introduction � la \textbf{th�orie du filtrage num�rique}.\newline
%
% Comme vous le verrez, cette premi�re version tente un parcours non-exhaustif des fondements de l'acoustique architecturale. La plupart des �l�ments de ce polycopi� ont �t� tir�s des cours personnels de \textbf{Manuel Melon} (Ma�tre de Conf�rences au CNAM) \cite{melon} pour les deux premiers chap�tres, de \textbf{Jean-Dominique Polack} (Professeur des Universit�s et Directeur du Laboratoire d'Acoustique Musicale de Paris VI) \cite{polack_salles} pour les deux chapitres suivants et du livre de \textbf{Jacques Jouhaneau} (Professeur titulaire de la Chaire d'Acoustique du CNAM) \cite{jouhaneau} pour de tr�s nombreuses r�f�rences. Je les remercie tous sinc�rement de leur aide.\newline

Ce document est en constante �volution. Certaines parties seront revues ou corrig�es dans les versions ult�rieures. Les mises � jour seront disponibles sur l'adresse web indiqu�e en page de garde et je vous invite donc � la consulter r�guli�rement.

Il est publi� selon les termes de la licence Creative Commons by-nc-sa 2.0 France :

\url{http://creativecommons.org/licenses/by-nc-sa/2.0/fr/}

\begin{figure}[b]
  \includegraphics[width=4cm]{img/Logo_cnam_new.jpg}
  \large{Copyright (C) 2006-2011 Guillaume Pellerin}
\end{figure}
